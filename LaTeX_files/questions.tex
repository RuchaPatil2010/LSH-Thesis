\documentclass{article}
\usepackage{my_style}
\usepackage{graphicx} % Required for inserting images
\usepackage{natbib}
\usepackage{}

\title{Questions}
\author{}
\date{}

\begin{document}

\maketitle

\begin{enumerate}
    \item As mentioned in Lemma 13, $T$ is any valid transformation of size $t$.
    A $T$ is valid for $x$ and $y$ only 
    if the number of $delete$ and $replace$ operations in $T\leq|x|$ and the 
    number of $insert$ and $replace$ operations in $T\leq|y|$.\\
    In the proof, it is mentioned that \textit{if $g(x,y,\rho)\in G_T$ and 
    $g(x,y,\rho)$ is complete, then $T$ is a prefix of $\mathcal{T}(x,y,\rho)$}.\\
    If we are given $x$, $y$, and $\rho$, then we can have only one gridwalk
    $g(x,y,\rho)$ which gives rise to only one transformation $\mathcal{T}(x,y,\rho)$,
    but we can have multiple valid $T$. How does the above statement hold true?

    \item In Lemma 13,
    while defining $g''$, the proof mentions: \textit{$g''$ consists of 
    loop and match operations concatenated onto the end of $g'\cdot\sigma$,
    ending with a match operation}. Consider a simple case where $h_\rho(x)=h_\rho(y)$,
    the last element in transcript would process \$ and have same value of $|s|$,
    hence the transcript could be hash-replace - hash-replace, in which case
    the Edit Distance operation would be replace, in this case $\sigma$ i.e.,
    the gridwalk would be of the form $g'\cdot\sigma$ and would not have the 
    $match$ operation mentioned above.

    \item The proof of Lemma 14 mentions: \textit{By Lemma 13, $h$ induces $T$ 
    on $x$ and $y$ (which is sufficient for $h(x)=h(y)$ by Lemma 12) with
    probability $p^r-2/n^2$}. Lemma 12 mentions the case where \textit{$T$ 
    \textbf{solves} $x$ and $y$} and the probability is derived from Lemma 13
    which mentions \textit{$T$ of length $t$ that is \textbf{valid} for $x$ 
    and $y$}.
\end{enumerate}

\bibliographystyle{alpha}
\bibliography{refs}



\end{document}
