\documentclass{article}
\usepackage{my_style}
\usepackage{graphicx} % Required for inserting images
\usepackage{natbib}
\usepackage{}

\title{Approximate Similarity Search Under Edit Distance Using Locality Sensitive Hashing}
\author{}
\date{}

\begin{document}

\maketitle

\subsection*{Algorithm}
We consider a random underlying function $\rho$ which takes an alphabet ($\Sigma$)
and the length of the output string ($|s|$) as the parameters and returns random
numbers $(r1,r2)$. $\therefore$ $\rho$ would have $|\Sigma|\times|s|$ keys which would
all return 2 random values in [0,1).\\
We randomly choose a value of $p\leq 1/3$.\\
$d$ is the length of the maximum string and $n$ is the total number of strings.
\\
\\
\textbf{How to Hash:}
\begin{enumerate}
    \item \textbf{Calculate $p_a$ and $p_r$:} \\
    $$p_a=\sqrt{\frac{p}{1+p}}$$
    $$p_r=\frac{\sqrt{p}}{\sqrt{1+p}-\sqrt{p}}$$
    \item \textbf{Initialise} $i=0$ and $s=""$.
    \item \textbf{Hashing:} while $i<|x|$ and $|s|<\frac{8d}{1-p_a}+6\log n$, we
    get the values $(r1,r2)$ from $\rho(x_i,|s|)$:
    \begin{enumerate}
        \item if $r1\leq p_a$, \textbf{hash-insert}: append $\bot$ to $s$
        \item if $r1>p_a$ and $r2\leq p_r$, \textbf{hash-replace}: append $\bot$ to 
        $s$ and increment $i$
        \item if $r1>p_a$ and $r2>p_r$, \textbf{hash-match}: append $x_i$ to $s$ and
        increment $i$.
    \end{enumerate}
\end{enumerate}

\subsection*{Analysis}

\textbf{Note:} we consider only the case where $h_\rho(x)=h_\rho(y)$ as they 
would belong to the same bucket. We ignore the cases where 
$h_\rho(x)\neq h_\rho(y)$, as the grid walk in this case would end up at STOP
node.\\
\textbf{Note:} if $h_\rho(x)=h_\rho(y)$, then, hash-match occurs only when
$x_{i_x}=y_{i_y}$ as hash-math inserts the actual alphabet (not $\bot$) in which
case both $\tau_k(x)$ and $\tau_k(y)$ would have same operation.\\
\\
For every value in transcript ($\tau$) of $x$ and $y$, we can define the edit
distance operation as:
\begin{center}
    \textbf{When $x_{i_x}\neq y_{i_y}$}\\
    \begin{tabular}{|c|c|c|}
        \hline
        $\tau_k(x)$ & $\tau_k(y)$ & ED Operation \\
        \hline
        hash-insert & hash-insert & loop \\
        hash-insert & hash-replace & insert \\
        hash-replace & hash-insert & delete \\
        hash-replace & hash-replace & replace \\
        \hline
    \end{tabular}\\
    
    \textbf{When $x_{i_x}= y_{i_y}$}\\
    \begin{tabular}{|c|c|c|}
        \hline
        $\tau_k(x)$ & $\tau_k(y)$ & ED Operation \\
        \hline
        hash-insert & hash-insert & loop \\
        hash-replace & hash-replace & match \\
        hash-match & hash-match & match\\
        \hline
    \end{tabular}
\end{center}

Hence, given any two string $x$ and $y$, if $h_\rho(x)=h_\rho(y)$, then we can
find a path from $(0,0)$ to $(|x|,|y|)$ in the Edit Distance table which can be
derived from the above table. And applying these operations on $x$, we would get
the string $y$.

\textbf{Note:} in \cite{McC21}, they have considered a "STOP" state in the Edit Distance 
table which is visited in case of inconsistency.\\


\subsection*{Lemmas and Proofs}

\begin{lem}[\textbf{Lemma 6}]
    For any string $x$ of length $d$,
    $\Pr_\rho  (\tau(x, \rho) \text{ is complete}) \geq 1-1/n^2$.\\
    \end{lem}
    
    \begin{proof}
    
    Recall that a transcript $\tau(x, \rho)$ is complete if
    $|\tau(x, \rho)| < 8d/(1-p_a)+6\log n$.
    If the transcript contains $l$ insert operations, $|\tau(x,\rho)| \leq d+l$
    since the maximum length of a string is $d$.
    
    \noindent
    We check the bounds for the probability of $l>7d/(1-p_a)+6\log n$.
    
    Consider success to be when we do not have hash-insert operations. This
    behaves like a geometric progression with probability = $(1-p_a)$. The
    expected number of hash-inserts we need to get an operation which is not a
    hash-insert would be $\frac{1}{1-p_a}$.
    Hence,
    \[E[l] = \frac{d}{1-p_a}\]
    
    \noindent
    The relevant Chernoff bound is:
    \begin{align}
    \Pr(X \geq (1+\delta)E[X]) & ~\leq ~( \frac{e^\delta}{(1+\delta)^(1+\delta)} )^{E[X]}  \nonumber \\
    & ~= ~e^{[\delta - (1+\delta)\ln(1+\delta)]*E[X]} \label{cbound}
    \end{align}
    
    \noindent
    We will be manipulating Equation (\ref{cbound}) in the sequel. \\
    To calculate $\Pr(l > 7d/(1-p_a) + 6\log n)$ with $E[X]=d/(1-p_a)$, 
    we set $\delta = 6 + \frac{6(1-p_a)\log n}{d}$.\\
    We know that
    $\delta - (1+\delta) \ln(1+\delta) \leq -\delta ^2/3 $, when $0 \leq \delta \leq 1$~\cite{HR90}.\\
    However, our value of $\delta = 6 + \frac{6(1-p_a)\log n}{d} > 6$, hence we cannot use the above bound. Instead, we use the fact that:
    \[\delta - (1+\delta)\ln(1+\delta) \leq -\delta/3\]
    when $\delta>1$.
    
    \noindent
    Substituting this in Equation (\ref{cbound}) and using the fact that $\frac{6d}{(1-p_a)} > 0$, we get:
    \begin{align}
    \Pr(l \geq (1+\delta)E[X]) & ~\leq ~e^{-\delta E[X]/3} \nonumber \\
    & ~<~ e^{-(6d/(1-p_a)+6\log n)/3}  \nonumber \\
    & ~<~ e^{-(6\log n)/3} \nonumber \\
    & ~=~ 1/n^2 \nonumber
    \end{align}
    Hence,
    $\Pr(l < 7d/(1-p_a) + 6\log n) > 1-1/n^2$\\
    
    \end{proof}  

\begin{lem}[\textbf{Lemma 7}]
    Consider a walk through $G(x,y)$ which at step $i$ takes the edge with label 
    corresponding to $g_i(x,y,\rho)$. Assume $k$ is such that the prefix 
    $g(x,y,\rho)[k]$ of length $k$ is alive. Then after $k$ steps, the walk 
    arrives at node $(i(x,k,\rho),i(y,k,\rho))$.
\end{lem}
\begin{proof}
    \textit{Intuitively:}\\
    If the gridwalk $g(x,y,\rho)$ is derived from transcript $\tau$, then at 
    the iteration $k$, the position would be at $(i(x,k,\rho),i(y,k,\rho))$, i.e.
    the respective pointers of $x$ and $y$ (provided that we do not enounter "STOP"
    in the process, i.e., the gridwalk is alive).
\end{proof}

\begin{lem}[\textbf{Lemma 8}]
    Let $x$ and $y$ be any two strings, and $\rho$ be any underlying function
    where both $\tau(x,\rho)$ and $\tau(y,\rho)$ are complete.\\
    Then $h_\rho(x)=h_\rho(y)$ if and only if $g(x,y,\rho)$ is alive. Furthermore,
    if $h_\rho(x)=h_\rho(y)$ then the path defined by $g(x,y,\rho)$ reaches
    node $(|x|,|y|)$.
\end{lem}
\begin{proof}
    \textit{Intuitively:}\\
    $g(x,y,\rho)$ is not alive only when it goes to the stop node, this happens
    only when one of the hash values is a hash-match and the other is not. As 
    hash-match records the value of $x_i$, and if $h_\rho(x)=h_\rho(y)$, then 
    both the hashes would have hash-match. 
    Also, if it is alive, as we end each string with \$, it has to go to the 
    very end, hence it will reach $(|x|,|y|)$ and the hash values would be equal.
\end{proof}

\begin{lem}[\textbf{Lemma 9}]
    Let $x$ and $y$ be any two strings, and for any $k<8d/(1-p_a)+6\log n$ Let
    $E_k$ be  the event that $i(x,k,\rho)<|x|$, $i(y,k,\rho)<|y|$, and 
    $x_{i(x,k,\rho)} \ne y_{i(y,k,\rho)}$. Then if $\Pr_\rho [E_k]>0$, the 
    following four conditional bounds hold:
    $$\Pr_\rho[g_k(x,y,\rho)=loop|E_k]=p_a^2$$
    $$\Pr_\rho[g_k(x,y,\rho)=delete|E_k]=p_a(1-p_a)p_r$$
    $$\Pr_\rho[g_k(x,y,\rho)=insert|E_k]=p_a(1-p_a)p_r$$
    $$\Pr_\rho[g_k(x,y,\rho)=replace|E_k]=(1-p_a)^2p_r^2$$
\end{lem}
\begin{proof}
    $$\Pr_\rho(\tau_k(x,\rho)=hash-insert|E_k)=p_a$$
    $$\Pr_\rho(\tau_k(x,\rho)=hash-replace|E_k)=(1-p_a)p_r$$
    $$\Pr_\rho(\tau_k(x,\rho)=hash-match|E_k)=(1-p_a)(1-p_r)$$
    Loop operation occurs when $\tau_k(x,\rho)$ is hash-insert and 
    $\tau_k(y,\rho)$ is hash-insert. Similarly, delete occurs when we have 
    hash-replace-hash-insert, insert occurs when we have hahd-insert-hash-replace
    and replace occurs when we have hash-replace-hash-replace.
    Multiplying the probabilities, we get the above values.\\
    \textbf{Note: }$p_r=p_a/(1-p_a)$, hence, $p_r(1-p_a)=p_a$. We can write 
    $p_a(1-p_a)p_r=p_a^2$ and $(1-p_a)^2p_r^2=p_a^2$. Substituting the values
    above, we get, $\Pr_\rho[g_k(x,y,\rho)=loop|E_k]=\Pr_\rho[g_k(x,y,\rho)=delete|E_k]
    =\Pr_\rho[g_k(x,y,\rho)=insert|E_k]=\Pr_\rho[g_k(x,y,\rho)=replace|E_k]=p_a^2$
\end{proof}

\begin{lem}[\textbf{Lemma 11}]
    Let $x$ and $y$ be two strings that do not contain \$. Then if $ED(x,y)=r$,
    \begin{enumerate}
        \item there exists a transformation $T$ of length $r$ that solves $x\cdot\$$
        and $y\cdot\$$
        \item there does not exist any transformation $T'$ of length $<r$ that 
        solves $x\cdot\$$ and $y\cdot\$$
    \end{enumerate}
\end{lem}
\begin{proof}
    ED can be transformed into transformation. \\
    ED of $x$ and $y$ is same as the ED of $x\cdot\$$ and $y\cdot\$$. AS ED is 
    the minimum distance between the strings, we cannot find a transformation
    of length less than that.
\end{proof}

\textbf{Note:} $T$ is the edit operations from the Edit Distance whereas $\mathcal{T}$
is obtained from the Grid Walk by removing loop and match.\\
For each transformation, we find the first index which differs from y and apply
the transformation there. We generate $r$ such strings in the process.\\

\begin{lem}[\textbf{Lemma 12}]
    Let $x$ and $y$ be two distinct strings and let $T=\mathcal{T}(x,y,\rho)$. Then 
    $h_\rho(x)=h_\rho(y)$ if and only if $T$ solves $x$ and $y$.
\end{lem}
\begin{proof}
    If $T=\mathcal{T}(x,y,\rho)$ solves $x$ and $y$, then the gridwalk $g(x,y,\rho)$
    starts at $(0,0)$ and ends at$(|x|,|y|)$, this is possible only when 
    $h_\rho(x)=h_\rho(y)$\\
    If $h_\rho(x)=h_\rho(y)$, then $\mathcal{T}(x,y,\rho)$ would be derived from
    the gridwalk $g(x,y,\rho)$ and hence it would solve $x$ and $y$.
\end{proof}

\begin{lem} [\textbf{Lemma 13}]
    For any $\$-terminal$ strings $x$ and $y$, let $T$ be a transformation of 
    length $t$ that is valid for $x$ and $y$. Then
    $$p^t-2/n^2 \leq \Pr_\rho(T\text{ is a prefix of }\mathcal{T}(x,y,\rho)) 
    \leq p^t$$
\end{lem}
\begin{proof}
    \begin{enumerate}
        \item \textbf{To prove $\Pr_\rho(T\text{ is a prefix of }\mathcal{T}(x,y
        ,\rho)) \leq p^t$:}\\
        Define $G_T$ as: all the transformations $T_g$ which contain $T$ as a 
        prefix, $G_T$ is the set of all the gridwalks such that $T_g$ are 
        derived from $G_T$.\\
        $\implies$ $G_T$ is alive(as $G_T$ does not contain STOP).\\
        To calculate the $\Pr$($g(x,y,\rho)\in G_T$)\\
        \textbf{Induction approach:}
        \begin{enumerate}
            \item \textbf{Base Step:} for $t=0$, all the transforms are valid, 
            hence, $\Pr(g(x,y,\rho)\in G_T)=1$.
            \item \textbf{Inductive Hypothesis: }$\sum\Pr_\rho(g(x,y,\rho)[t-1]
            \text{ is a prefix of }G_{T'})=p^{(t-1)}$ \textit{where $G_{T'}$ is 
            a set of all the transformations $T_g$ with the last operation 
            removed.}
            \item \textbf{Inductive Step: }\\
            Let the last operation be $\sigma$.\\
            $g(x,y,\rho)=g'(x,y,\rho)\cdot \sigma\cdot\{loop,match\}^*\cdot
            \{loop\}^*$\\
            Define: \\
            $g''=g'(x,y,\rho)\cdot \sigma$,\\
            $g'''=g''\cdot \{loop,match\}^*$,\\
            $g(x,y,\rho) = g'''\cdot\{loop\}^*$\\
            The Probability can be defined as:\\
            $\Pr_\rho(g(x,y,\rho)\in G_T)=\Pr_\rho(g'(x,y,\rho)\in G_{T'})\cdot
            \Pr(g''\in G_T|g'(x,y,\rho)\in G_{T'})\cdot \Pr(g'''\in G_T|g''\in 
            G_T)\cdot\Pr_\rho(g(x,y,\rho)\in G_T|g'''\in G_T)$\\
            \textbf{Note:} we can directly multiply the probabilities as all the
            values are subset of the respective conditions.\\
            Now,\\
            $\Pr_\rho(g'(x,y,\rho)\in G_{T'})=p^{t-1}$ --- \textit{from Inductive 
            hypothesis}\\
            $\Pr(g''\in G_T|g'(x,y,\rho)\in G_{T'})=p_a^2$ --- \textit{as from 
            Lemma 9, all the operations have a probability of $p_a^2$.}\\
            $\Pr(g'''\in G_T|g''\in G_T)=1$ --- \textit{as $G_T$ does not 
            contain "STOP" and we are done with the last operation, the only 
            option left is "LOOP" or "MATCH" if $x_{i_x}=y_{i_y}$ then we have 
            "MATCH", o.w. "LOOP"}. \\
            $\Pr_\rho(g(x,y,\rho)\in G_T|g'''\in G_T)=\frac{1}{1-p_a^2}$ ---
            \textit{as addind loop operations is like a Poisson distribution 
            with the probability of success=$1-p_a^2$}\\
            Multiplying all, we get:
            $\Pr_\rho(g(x,y,\rho)\in G_T)=p^{t-1}\frac{p_a^2}{1-p_a^2}=p^t$            

        \end{enumerate}
    
        \item To prove $p^t-2/n^2 \leq \Pr_\rho(T\text{ is a prefix of }\mathcal{T}(x,y,\rho))$:\\
    $T$ is a prefix of $\mathcal{T}(x,y,\rho)$ if $g(x,y,\rho)\in G_T$ and $g(x,y,\rho)$
    is complete, i.e., the transcripts $\tau(x,\rho)$ and $\tau(y,\rho)$ are complete:
    \begin{align}
        \Pr(T\text{ is a prefix of }\mathcal{T}(x,y,\rho))=&\Pr(g(x,y,\rho)\in G_T)
    +\Pr(g(x,y,\rho)\text{ is complete}) \nonumber\\
    &-\Pr(g(x,y,\rho)\in G_T \cap g(x,y,\rho)\text{ is complete})\nonumber
    \end{align}
    \begin{align}
        \Pr(g(x,y,\rho)\in G_T)&=p^t\text{---from point 1} \nonumber\\
        \Pr(g(x,y,\rho)\text{ is complete})&=1-\Pr(g(x,y,\rho)\text{ is not complete})\nonumber\\
        &=1-\Pr(\tau(x,\rho)\text{ is not complete}|\tau(y,\rho)\text{ is not complete})\nonumber\\
        &\geq 1-(\Pr(\tau(x,\rho)\text{ is complete})
    +\Pr(\tau(y,\rho)\text{ is complete}))\nonumber\\
    &\text{\hspace{1cm}    ---using union bound}\nonumber\\
        &=1-(\frac{1}{n^2}+\frac{1}{n^2})\text{  ---from Lemma 6}\nonumber\\
        &=1-\frac{2}{n^2}\nonumber     
    \end{align}
    \begin{align}
        \Pr(g(x,y,\rho)\in G_T \cap g(x,y,\rho)\text{ is complete})&\leq 1\nonumber\\
        -\Pr(g(x,y,\rho)\in G_T \cap g(x,y,\rho)\text{ is complete})&\geq -1 \nonumber
    \end{align}
    Adding all, we get:
    $$\Pr(T\text{ is a prefix of }\mathcal{T}(x,y,\rho))\geq p^t-\frac{2}{n^2}$$
    \end{enumerate}
\end{proof}

\textbf{Bounds on Collision Probability:}
\begin{lem}[\textit{Combining }\textbf{Lemma 14 }\textit{and }\textbf{Lemma 15}:]
    If $x$ and $y$ satisfy $ED(x,y)\leq r$, then $\Pr_\rho(h_\rho(x)=h_\rho(y))
    \geq p^r-\frac{2}{n^2}$.\\
    If $x$ and $y$ satisfy $ED(x,y)\geq cr$, then $\Pr_\rho(h_\rho(x)=h_\rho(y))
    \leq (3p)^{cr}$.
\end{lem}
\begin{proof}
    We get the lower bound from \textbf{Lemma 13}, if $ED(x,y)\leq r$, then there
    exists a transformation $T$ of size less than $r$. The probability that 
    $T$ is a prefix of $\mathcal{T}(x,y,\rho)\geq p^t-\frac{2}{n^2}$.\\
    For the upper bound, let $\mathcal{T}$ be the set of all the transformations that 
    solve $x$ and $y$, then $\Pr_{h\in H}(h(x)=h(y))=\sum_{T\in\mathcal{T}}^{}p^{|T|}$.\\
    We can imagine the transformations in form of a trie with each node having 3 
    children for insert delete and replace and the minimum depth of any leaf is 
    $cr$.\\
    For all the nodes with $depth>cr$, we merge the children into the parent node,
    so the probability value which was earlier $3p^i$ for the 3 child nodes would 
    now be $p^{i-1}$ which is obviously greater as $p<1/3$. Hence,
    \begin{align}
        \Pr_{h\in H}(h(x)=h(y))&=\sum_{T\in\mathcal{T}}^{}p^{|T|}\nonumber\\
        &\leq \sum_{T\in\mathcal{T}}^{}p^{cr}\nonumber\\
        &=(3p)^{cr}\text{---as we have }3^{cr}\text{ leaves}\nonumber
    \end{align}
\end{proof}

\bibliographystyle{alpha}
\bibliography{refs}



\end{document}
