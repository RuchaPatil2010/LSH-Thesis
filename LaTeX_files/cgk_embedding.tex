\documentclass{article}
\usepackage{my_style}
\usepackage{graphicx} % Required for inserting images
\usepackage{natbib}

\title{Review on CGK Embedding}
\author{}
\date{}

\begin{document}

\maketitle

\noindent 
The paper describes two methods: 

\begin{enumerate}
    \item [Embedding] This embeds the 2 strings $x$ and $y$ into $f(x,r)$ and 
    $f(y,r)$ using a random string $r$ which creates $3n$ hash functions 
    $h_1,h_2,...,h_{3n}: \{0,1\} \rightarrow \{0,1\}$.
    $\therefore f(x,r): \{0,1\}^n \times \{0,1\}^{6n} \rightarrow \{0,1\}^{3n}$.
    We compare and get the Hamming distance of $f(x,r)$ and $f(y,r)$ and claim
    that:
    $$\frac{1}{2}\cdot \Delta_e(x,y) \leq \Delta_H(f(x,r),f(y,r)) \leq
    O((\Delta_e(x,y))^2)$$
    with probability atleast 2/3.

    \item [Kernelization] This converts $x$ and $y$ into $x'$ and $y'$ such 
    that $\Delta_e(x,y)=\Delta_e(x',y')$. It uses the following 2 methods:
    \begin{enumerate}
        \item [Deflation] We increase the length of $x$ and $y$ and keep the 
        EditDistance same. We can prove that there exists a substring $w$ 
        in $x$ and $y$ of the form $w=p^r$ where $p$ is the periodicity of $w$
        and $r>2$. Somewhere in $w$, the strings $x$ and $y$ would align wrt
        EditDistance, as $w$ is same in both the strings $x$ and $y$, all the 
        bits after would be aligned too. We can add $p$ sometime after the initial 
        alignment index such that $w'=p^{r+1}$ and $\Delta_e(x,y)=\Delta_e(x',y')$

        \item [Shrinkage] Similar to the above case, we can remove $p$ bits form
        $w$ and the EditDistance would remain the same. In case of 
        \textbf{Shrinkage}, we define $s=K+2k+(k+1)(t+1)$ and reduce $w$ by
        keeping the first $s$ and last $s$ bits.
    \end{enumerate}
    Decompose $x=u_0w_1u_1...w_lu_l$ and $y=v_0w_1v_1...w_lv_l$, deflate and 
    shrink each $w_i$ to get the desired $x'$ and $y'$.
\end{enumerate}

\noindent
\textbf{Note:} this paper only compares 2 binary strings with low edit distance
$O(n^{1/6})$, where $n$ is the length of both the strings.

\pagebreak
\section*{Theorems and Proofs}
\begin{theorem}[\textbf{Theorem 4} in \cite{CGK16}] 
    The mapping $f:\{0,1\}^n \times \{0,1\}^{6n} \rightarrow \{0,1\}^{3n}$ 
    computed by Algorithm 1 satisfies the following condtions:
    \begin{enumerate}
        \item For every $x \in \{0,1\}^n$, given $f(x,r)$ and $r$, it is possible
        to decode back $x$ with probability $1-exp(-\Omega(n))$
        \item For every $x,y \in \{0,1\}^n$, $\Delta_e(x,y)/2 \leq \Delta_H(f(x,r),f(y,r))$
        with probability at least  $1-exp(-\Omega(n))$
        \item For every positive constant $c$ and every $x,y \in \{0,1\}^n$, 
        $\Delta_H(f(x,r),f(y,r))\leq c \cdot(\Delta_e(x,y))^2$ with probability
        at least $1-\frac{12}{\sqrt{c}}$
    \end{enumerate}
\end{theorem}

\begin{proof}[\textbf{Proof}]
    \begin{enumerate}
        \item we can decode back $x$ if we are given $f(x,r)$ and $r$ only if 
        the value of $i$ from Algorithm 1 is $n+1$ at the end of $f(x,r)$. As
        this would mean that we have traversed through all the bits of $x$.\\
        Consider that we have infinite hash functions 
        $h_1,h_2,...:\{0,1\}\rightarrow\{0,1\}$.\\
        Consider for each $x_i$ we embed it using hash functions $h_k,...,h_l$.
        As we embed $x_i$ for all $h_k,...,h_l$, it implies that 
        $h_k(x_i)=...=h_{l-1}(x_i)=0$ and $h_l(x_i)=1$.\\
        Hence, we can interpret the given condition as $n$ geometric distributions,
        where the total number of trials required is less than $3n$.\\
        \\
        For every grometric distribution, $p=1/2$.\\
        Define, $X_i = \text{Number of hash functions used to embed }x_i$, and
        all the $X_i$ are i.i.d\\
        \begin{align}
            E[X_i] &~=~ 2 \nonumber \\
            \therefore E[X] &~=~ 2n \nonumber        
        \end{align}
        Using Equation (6) from \cite{HR90}
        \begin{align}
            \Pr(S\geq(1+\epsilon)m) &~\leq ~e^{-\epsilon^2m/3} \nonumber\\
            (1+\epsilon)&~=~3/2 \nonumber\\
            \epsilon&~=~1/2 \nonumber\\
            \Pr(X>3n)&~\leq ~e^{-\frac{(1/2)^22n}{3}} \nonumber\\
            &~\leq ~e^{-n/6} \nonumber \\
            \therefore \Pr(X<3n) &~\geq ~1-e^{-n/6} \nonumber
        \end{align}
        
        \item Consider the $i$ value mentioned in Algorithm 1 is $n+1$ for both
        $x$ and $y$.\\
        Let $l = \Delta_H(f(x,r),f(y,r))$, then we need to apply atmost $l$ edit
        operations to $x$ to get $y$. \\
        Except, when atmost the last $l$ bits of $y$ are 0 and align with the 
        padded 0s of $x$. (The paper \cite{CGK16}, does not mention the last 
        bits of $x$ here, but I think that maximum of $l$ bits from either $x$ and
        $y$ could be aligned with the padded 0s of the other).\\
        Hence, $\Delta_e(x,y) \leq 2l$.\\
        As per out initial assumption, this is possible only when the $i$ value
        reach $n+1$ for both $x$ and $y$.\\
        \begin{align}
            \Pr(X<3n \cap Y<3n) &~=~ \Pr(X<3n) \cdot \Pr(Y<3n) ---(X and Y are independent events) \nonumber \\
            &~=~ (1-e^{-n/6})\cdot(1-e^{-n/6}) \nonumber\\
            &~=~ 1-2e^{-n/6}+e^{-n/3} \nonumber\\
            &~\approx~ 1-2e^{-n/6} \nonumber \\
            &~=~ 1-e^{-(n/6-log2)} \nonumber \\
            &~=~ 1-e^{(-\Omega(n))} \nonumber
        \end{align}
        \item This can be proved by combining \textbf{Lemma 4.2} and 
        \textbf{Proposition 3.2}.\\
        \textbf{Lemma 4.2}: 
        \begin{align}
            \Pr(\Delta_H(f(x,r),f(y,r))\leq l) &~\geq~ 
            \sum_{t=0}^{l} q(t,\Delta_e(x,y))\nonumber\\
            \text{For our case, $l=c\cdot (\Delta_e(x,y))^2$} \nonumber\\
            \Pr(\Delta_H(f(x,r),f(y,r))\leq c\cdot (\Delta_e(x,y))^2) &~\geq~ 
            \sum_{t=0}^{c\cdot (\Delta_e(x,y))^2} q(t,\Delta_e(x,y)) \nonumber\\
            \text{From \textbf{Proposition 3.2},} 
            \sum_{t=0}^{l} q(t,k) \geq 1- \frac{12k}{\sqrt{l}} \nonumber \\
            \Pr(\Delta_H(f(x,r),f(y,r))\leq c\cdot (\Delta_e(x,y))^2) &~\geq~ 
            1-\frac{12\Delta_e(x,y)}{\sqrt{c\cdot (\Delta_e(x,y))^2}}  \nonumber \\
            &~\geq~ 1-\frac{12}{\sqrt{c}} \nonumber
        \end{align}
    \end{enumerate}
\end{proof}

\begin{lem}[\textbf{Lemma 4.2} in \cite{CGK16}] 
    Let $x,y \in \{0,1\}^n$ be of edit distance $\Delta_e(x,y) = k$. Let $q(t,k)$
    be the probability that a random walk on the integer line starting from the 
    origin visits the point $k$ at time $t$ for the first time. Then for any
    $l>0$, $\Pr(\Delta_H(f(x,r),f(x,y)) \leq l) \geq \sum_{t=0}^{l} q(t,k)$ 
    where the probability is over the choice of $r$.
\end{lem}

\begin{proof}[\textbf{Proof}][\textit{My interpretation}]
    $k$ is the Edit Distance $\Delta_e(x,y)$.\\
    Imagine a timeline, we start at 0 and have a marker at $k$.\\
    Our methodology for moving on the timeline is as follows:
    \begin{enumerate}
        \item $i_x(t)$ and $i_y(t)$ are the indices of $x$ and $y$ being embedded
        at time $t$. Our position on the timeline would be $i_x(t)-i_y(t)$ at any
        given time. Since the Edit Distance = $k$, the value of $i_x(t)-i_y(t)$
        should be less than $k$ at all times.
        \item Let $d_t=i_x(t)-i_y(t)$,
        \begin{enumerate}
            \item when $x_{ix(t)}=y_{iy(t)}$, then 
            $h_t(x_{ix(t)}),h_t(y_{iy(t)})=(0,0) \text{ or }(1,1) $\\
            $\implies $ value of $d_t$ does not change.\\

            \item when $x_{ix(t)}\neq y_{iy(t)}$, then 
            $h_t(x_{ix(t)}),h_t(y_{iy(t)})=(0,0),(0,1),(1,0),(1,1) $\\
            $\implies$ value of $d_t$ changes only when $x_{ix(t)} \neq y_{iy(t)}$ \\
            $\implies$ $d_t$ changes only when it is contributing to the Hamming Distance.
            $\implies$ Hamming Distance could be interpreted as our movement on the
            timeline.
        \end{enumerate}
        \item Ignoring the steps when $x_{ix(t)}=y_{iy(t)}$, the probability that
        we move $+1$ steps is $1/4$, $-1$ steps is $1/4$ and stay where we are
        is $1/2$. 
        \item Hence, the \textbf{Lemma} statement can be interpreted as: The Probability 
        that we take atmost $l$ steps to reach $k$ is greater than the probability 
        that we reach $k$ for the first time within $l$ steps.
    \end{enumerate}
\end{proof}


\bibliographystyle{alpha}
\bibliography{refs}

\end{document}
